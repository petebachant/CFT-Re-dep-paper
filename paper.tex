%  LaTeX support: latex@mdpi.com
%  In case you need support, please attach any log files that you could have,
% and specify the details of your LaTeX setup (which operating system and LaTeX
% version / tools you are using).

%=================================================================

% LaTeX Class File and Rendering Mode (choose one)
% You will need to save the "mdpi.cls" and "mdpi.bst" files into the same folder
% as this template file.

%=================================================================

\documentclass[energies,article,accept,moreauthors,pdftex,12pt,a4paper]{mdpi}

%=================================================================
\setcounter{page}{1}
\lastpage{x}
\doinum{10.3390/------}
\pubvolume{xx}
\pubyear{2015}
\history{Received: xx / Accepted: xx / Published: xx}
%------------------------------------------------------------------
% The following line should be uncommented if the LaTeX file is uploaded to
% arXiv.org
%\pdfoutput=1

%=================================================================

% Add packages and commands to include here
% The amsmath, amsthm, amssymb, hyperref, caption, float and color packages are
% loaded by the MDPI class.
\usepackage{graphicx}
%\usepackage{subfigure,psfig}
\usepackage[draft]{todonotes}

\def \p{\partial}
\def \d{\mathrm{d}}
\def \D{\mathrm{D}}

%=================================================================
%% Please use the following mathematics environments:
%\theoremstyle{mdpi}
%\newcounter{thm}
%\setcounter{thm}{0}
%\newcounter{ex}
%\setcounter{ex}{0}
%\newcounter{re}
%\setcounter{re}{0}
%\newtheorem{Theorem}[thm]{Theorem}
%\newtheorem{Lemma}[thm]{Lemma}
%\newtheorem{Characterization}[thm]{Characterization}
%\newtheorem{Proposition}[thm]{Proposition}
%\newtheorem{Property}[thm]{Property}
%\newtheorem{Problem}[thm]{Problem}
%\newtheorem{Example}[ex]{Example}
%\newtheorem{Remark}[re]{Remark}
%\newtheorem{Corollary}[thm]{Corollary}
%\newtheorem{Definition}[thm]{Definition}
%% For proofs, please use the proof environment (the amsthm package is loaded by
% the MDPI class).

%=================================================================

% Full title of the paper (Capitalized)
\Title{Effects of Reynolds Number on the Performance and Near-Wake of a
Vertical-Axis Cross-Flow Turbine}

% Authors (Add full first names)
\Author{Peter Bachant $^{1,}$* and Martin Wosnik $^{1}$}

% Affiliations / Addresses (Add [1] after \address if there is only one
% affiliation.)
\address{%
$^{1}$ Center for Ocean Renewable Energy, University of New Hampshire, 24
Colovos Rd., Durham, NH, USA}

% Contact information of the corresponding author (Add [2] after \corres if
% there are more than one corresponding author.)
\corres{pxL3@unh.edu}

% Abstract (Less than 200 words)
\abstract{The acceptable Reynolds number mismatch for scaled physical model
testing a cross-flow turbine is investigated experimentally, both with respect
to prediction of full-scale performance, and wake characteristics, which will be
drivers of overall performance and interaction between multiple turbines.}

% Keywords: add 3 to 10 keywords
\keyword{keyword; keyword; keyword}

% The fields PACS, MSC, and JEL may be left empty or commented out if not
% applicable
%\PACS{}
%\MSC{}
%\JEL{}

\begin{document}

\listoftodos

\section{Introduction}

Scaled physical models are often used in science and engineering to approximate
real-world systems. However, the principle of dynamic similarity allows for
geometrically scaled systems to be dynamically identical in scale if certain
nondimensional physical parameters are matched. In the case of fluid systems,
the most important parameter is often the Reynolds number, which quantifies the
importance of viscosity on the flow. As such, hereafter we will use the word
``scale'' to refer to this dynamical scale rather than the geometric scale.

With regards to wind and marine hydrokinetic (MHK) turbines, scaled physical
models are used to validate predictive numerical models, predict full-scale
performance of individual turbines, and design or investigate arrays of devices.
Despite being significantly less expensive, there is still some concern at which
scale an experiment must be performed to be relevant to full-scale application.

For numerical models, one might obtain favorable predictions for scaled systems
due to modified physics. If the numerical model is then considered
``validated,'' there is a risk that its application at full-scale may produce
incorrect predictions. For scaled physical prototypes, it is of interest how
small---since size, or scale generally correlates inversely with cost---the
prototype can be while remaining a reliable predictor of full-scale behavior. At
the very least, it is necessary to test at a scale where results can be
extrapolated, i.e. changes are occurring quasi-linearly without dramatic changes
in the underlying physical principles governing the behavior of interest, e.g.
power production in the case of a wind or MHK turbine.

Blackwell et al. investigated the effects of Reynolds number on the performance
of a 2 m diameter Darrieus vertical-axis cross-flow wind turbine with NACA 0012
blades \cite{Blackwell1976}. By varying wind tunnel speed, the turbine was made
to operate at approximately constant blade chord Reynolds number $Re_c$ ranging
from $1 \times 10^5$ to $3 \times 10^5$. In this regime the turbine power
coefficient $C_P$ was shown to be sensitive to $Re_c$, with sensitivity
diminishing at the higher Reynolds numbers, especially for turbines of lower
solidity ($Nc/R$, where $N$ is the number of blades and $R$ is the turbine's
maximum radius). More recently, Bravo et al. observed power coefficient of a
straight-bladed VAWT to become Reynolds number independent at $Re_c \approx 4
\times 10^5$.

When designing or studying arrays, it is common to use very small
(geometrically) scaled devices. 
% %
\todo[inline]{Get a reference for this, and a sample geometric scale.} 
% %
Sometimes these are not even models of turbines, but
wake generating objects, e.g., porous disks, that are meant to replicate the
wakes of real turbines. In this case, it is of interest to determine at what
scale one might be able to realistically study wake flows in an array, and also
to evaluate the effectiveness of a wake generator. In other words, a wake
generator may do a fine job simulating a scaled turbine, but how well can it
simulate a real turbine?

Previously, Bachant and Wosnik \cite{Bachant2015-JoT} showed that a cross-flow
turbine wake's unique vortical mean flow field is responsible for accelerated
wake recovery when compared with conventional axial-flow propeller-type
turbines. In this study we seek to replicate the same energy balance
considerations at multiple Reynolds numbers, to examine the implications on how
small scale, i.e. low Reynolds number experiments may be used to study flows in
turbine arrays.

Bachant and Wosnik observed how maximum performance and 2-D near-wake profiles
are affected by Reynolds number, but here we look deeper. In this study we set
out to measure the effects of scale, i.e. Reynolds number, on the performance of
the turbine, and its more detailed near-wake characteristics. We were looking to
find a threshold scale beyond which a physical model study could reliably
predict full-scale performance. We were also looking to investigate the
sensitivity as small scales, and what the implications might be for validating
the prediction capabilities of numerical models.

\todo[inline]{Talk about interdependence of Froude number.}

\subsection{Reynolds Number Effects}

In this study we are focused on a turbine constructed from foils---a lift-driven
device in contrast to drag-driven. A logical starting point for understanding
the effects of $Re$ on overall turbine loading is to look at simplified cases
of individual foils, with respect to their static and dynamic loading. 

\subsubsection{On Static Foil Behavior}

Turbine performance is strongly dependent on the lift-to-drag ratio $l/d$ acting
on the blades, which is a function of the blades' angles of attack $\alpha$.
These angles of attack are a function of turbine tip speed ratio, induction
(slowing/turning of the free stream flow prior to reaching the turbine), and
turbine rotation or azimuthal angle $\theta$. Looking at static airfoil data, we
see that the static stall angle---the angle just beyond that at which $l/d$ is
maximum, characterized by flow separation near the leading edge---increases with
blade chord Reynolds number $Re_c$. This can be explained as the transition to a
turbulent boundary layer adjacent to the blade delaying separation. 

As previously mentioned, the near-wake of a CFT is not very similar to a simple
drag source, or plane shear flow. One of the mean wake's streamwise vortex pairs
can be predicted by potential flow theory as a result of finite-span blades
producing lift (blade end effects), but the other---caused by viscous effects,
i.e., dynamic stall---is not. We can attempt to predict changes
\todo[inline]{Finish this thought with Reynolds number dependence of these
simpler flows, and set up the question whether or not we can assume these will
combine linearly to produce the wake $Re$-dependence we see.}

\begin{figure}[ht]
\caption{Airfoil lift to drag ratio as a function of Reynolds number, from 
\cite{McMasters1980}.}
\end{figure}

\subsubsection{On Dynamic Foil Behavior}

Static foil performance does not tell the whole story for a cross-flow turbine.
The azimuthal, and therefore temporal variation of $\alpha$ in a cross-flow
turbine implies the occurrence of dynamic stall at tip speed ratios at and below
those of maximum rotor torque\cite{Para2002}. We then must understand the
effects of Reynolds number on the dynamic stall process.

Dynamic effects on blade loading can be distinguished between attached and
separated conditions. 

Bousman showed that dynamic stall was insensitive to Reynolds number for $Re=1.0
\times 10^5$--$2.5 \times 10^5$ \cite{Bousman2000-evaluation}.

\subsubsection{On Wake Flows}

We expect that lower lift on the blades at lower $Re$ will weaken tip vortex
shedding, and decrease the levels of turbulence shed from the blade boundary
layers. As mentioned previously, the dynamic stall vortex shedding is not
expected to have a large Reynolds number sensitivity.

\todo[inline]{What do studies in the literature tell us about $Re$ effects
on shear flows, wakes, and vortex flows?}

Chamorro et al. showed that turbulence statistics in an AFT wake became
$Re$-independent at $Re_D \approx 1 \times 10^5$, with mean velocity slightly
earlier at $Re_D \approx 5 \times 10^4$ \cite{Chamorro2012}.

\section{Experimental Setup}

Experiments were performed in the University of New Hampshire's tow/wave tank
turbine test bed. The turbine model is dubbed the UNH-RVAT, for Reference
Vertical Axis Turbine, which is designed to be a generic case for numerical
model testing, similar to the Sandia National Labs/US Department of Energy RM2
River Turbine, but with higher solidity, or blade chord to radius ratio. Details
of the turbine and experimental setup are described in \cite{Bachant2015-JoT},
and a CAD model of the turbine is available from \cite{Bachant2014-RVAT-CAD}.

\begin{figure}[ht!]
\centering
\includegraphics[width=0.75\textwidth]{figures/exp_setup_drawing}
\caption{Drawing of the experimental setup.}
\label{fig:exp-setup}
\end{figure}

\todo[inline]{Make sure to use drawing of turbine with no hubs, since these were
included in the tare drag}


\subsection{Test Plan} 

Approximately 1500 tows were performed, each one used for a single data point on
either a performance curve or wake map. A performance curve consisted of 31
tows, where during each tow the mean turbine tip speed ratio was held constant,
ranging from 0.1--3.1 in 0.1 increments. Full performance curve data were
acquired for tow speeds from 0.4 to 1.2 m/s in 0.2 m/s increments, for which
turbine diameter and approximate blade chord Reynolds number are presented in
Table~\ref{tab:Re}. Performance was also measured for $\lambda=1.9$ at tow
speeds $[0.3, 0.5, 0.7, 0.9, 1.1, 1.3]$ m/s for two tows each.

\begin{table}
\centering
\begin{tabular}{ccc}
Tow speed (m/s) & $Re_D$ & $Re_{c,\mathrm{ave}}$ ($\lambda = 1.9$) \\ 
\hline
0.3 & $3.0 \times 10^5$ & $8.0 \times 10^4$ \\ 
0.4 & $4.0 \times 10^5$ & $1.1 \times 10^5$ \\ 
0.5 & $5.0 \times 10^5$ & $1.3 \times 10^5$ \\ 
0.6 & $6.0 \times 10^5$ & $1.6 \times 10^5$ \\ 
0.7 & $7.0 \times 10^5$ & $1.9 \times 10^5$ \\ 
0.8 & $0.8 \times 10^5$ & $2.1 \times 10^5$ \\ 
0.9 & $0.9 \times 10^5$ & $2.4 \times 10^5$ \\ 
1.0 & $1.0 \times 10^6$ & $2.7 \times 10^5$ \\ 
1.1 & $1.1 \times 10^6$ & $2.9 \times 10^5$ \\ 
1.2 & $1.2 \times 10^6$ & $3.2 \times 10^5$ \\ 
1.3 & $1.3 \times 10^6$ & $3.4 \times 10^5$ \\ 
\end{tabular} 
\caption{Turbine diameter and approximate blade chord Reynolds numbers for the
tow speeds used in the experiment.}
\label{tab:Re}
\end{table}

A wake map was generated by positioning a Nortek Vectrino+ acoustic Doppler
Velocimeter (ADV) at 270 different locations, varied in the cross-stream and
vertical directions at one turbine diameter downstream ($x/D=1$). These
locations have vertical coordinates from the turbine centerline up to
$z/H=0.625$, ranging in the cross-stream direction $y/R = \pm 3$. These
locations are shown in Figure~\ref{fig:wake-locations}.

\begin{figure}
\centering
\includegraphics[width=0.9\textwidth]{figures/turbine_coordinate_system}
\caption{Wake measurement coordinate system and locations. Dimensions are in
meters.} 
\label{fig:wake-locations}
\end{figure}

\subsection{Data Processing}

From each set of tows, a standard time interval was set, and each run was
analyzed to compute statistics over this interval, truncating the end slightly
to achieve and integer number of blade passages.

\todo[inline]{Cite reduced data and processing code on figshare.}


\section{Results and Discussion}


\subsection{Performance}

Full performance curves for various Reynolds numbers are plotted in
Figure~\ref{fig:perf-curves}. There is essentially no change in the shape of the
drag coefficient curves---merely an upward shift in $C_D$ with increasing $Re$.

In general, maximum $C_P$ increases with Reynolds number, which is due to an
increase in the foil lift-to-drag ratio. The power coefficient curves also show
a downward shift in the optimal tip speed ratio with increasing Reynolds number.
This is caused by the stall delay from a more turbulent boundary later on the
blade suction side.

\begin{figure}[ht]
\includegraphics[width=0.95\textwidth]{figures/perf_curves}
\caption{Mean power (left) and drag (right) coefficient curves plotted for
multiple Reynolds numbers.}
\label{fig:perf-curves}
\end{figure}

Mean power and drag coefficients at $\lambda=1.9$ are plotted for all Reynolds
numbers in Figure~\ref{fig:perf-Re-dep}. There is a drastic improvement in $C_P$
with increasing Reynolds number for the lower values. Power coefficient then
becomes  essentially $Re$-independent at $Re_D = 0.8 \times 10^6$, which
corresponds to an approximate average blade chord Reynolds number $Re_{c,
\mathrm{ave}} = 2.1 \times 10^5$. This threshold is consistent with the behavior
of the blade boundary layer transitioning from laminar to turbulent, thereby
promoting reattachment of the laminar separation bubble.

The behavior of the mean rotor drag coefficient is similar, though the changes
are less dramatic. This is likely due to cross-flow turbine geometry, where
increases in blade drag at lower $Re$ somewhat offset the reduction in lift, to
keep total streamwise force from changing as much as the rotor torque. The
tendency for $C_D$ to continue increasing with $Re$ may also be a consequence of
increasing Froude number, which therefore increases free surface deformation and
wave drag during towing without increasing flow through the turbine.

\begin{figure}[ht]
\includegraphics[width=0.95\textwidth]{figures/perf_re_dep}
\caption{Mean power (left) and drag (right) coefficients at $\lambda=1.9$
plotted versus Reynolds number.} 
\label{fig:perf-Re-dep}
\end{figure}

\subsubsection{Relation to Static Foil Characteristics}

To help understand---and possibly predict---the $Re$-sensitivity on turbine
performance, a series of static foil coefficient datasets were computed with the
XFOIL viscous panel method code \cite{Drela1989}, implemented as part of the
open source turbine design software QBlade. Simulations were run with the
default settings and zero Mach number, for the approximate average blade chord
Reynolds numbers encountered in the experiments.

To investigate the effects of the blades' ``virtual camber'' due to their
circular path \cite{Migliore1980}, the XFOIL calculations were performed for
20\% thick foils with 0\% (NACA 0020), 2\% (NACA 2520), and 4\% (NACA 4520)
camber about their half-chord location. A 4\% camber approximates the maximum
distance between the blade chord line and path for the UNH-RVAT and the 2\%
camber takes into account the reduction in virtual flow curvature from the
non-curved inflow velocity by dividing by the tip speed ratio $\lambda=1.9$.

Results from the XFOIL calculations are shown in Figure~\ref{fig:foil-Re-dep},
where values of maximum lift coefficient, minimum drag coefficient, and maximum
lift-to-drag ratio are normalized (for comparison between foils) and plotted
versus $Re_c$. In general, larger camber is associated with decreased foil
performance at lower Reynolds number.

\begin{figure}[ht!]
\centering
\includegraphics[width=0.95\textwidth]{figures/all_foils_re_dep}
\caption{Foil section characteristics computed by XFOIL for various $Re_c$.}
\label{fig:foil-Re-dep}
\end{figure}

In conjunction with the cross-flow turbine blade kinematics, the foil
coefficients were used to approximate turbine performance by calculating the
peak torque coefficient on the upstream half of the blade path. The turbine
torque coefficient $C_T$ can be related to the blade section chordwise force
coefficient $C_c$ by
\begin{equation}
C_T = \frac{C_c c}{2R} \frac{|U_\mathrm{rel}|^2}{U_\infty^2},
\end{equation}
where the blade section chordwise force (for zero preset pitch)
\begin{equation}
C_c = C_l \sin \alpha - C_d \cos \alpha.
\end{equation}

The relative blade velocity was calculated by adding the free stream to the
opposite of the blade tangential velocity. Note that this neglects any
induction, or slowing of the free stream by the turbine forces, which would be
present in a momentum/streamtube model. Since we were not looking to predict
absolute performance but rather relative changes with $Re$, this method was
deemed acceptable as it is extremely simple and fast to compute. 

Values for the blade angle of attack, relative velocity, and torque coefficient
are plotted in Figure~\ref{fig:blade-kinematics}. The effects of static stall
are clearly present in the torque coefficient plot, and by the time the angle of
attack has decreased below that of stall, the relative velocity is so low that
there is not much contribution to the torque coefficient.

\begin{figure}[ht!]
\centering
\includegraphics[width=0.95\textwidth]{figures/foil_kinematics_ct}
\caption{Geometric angle of attack (a), relative velocity (b) and torque coefficient
(c) calculated with a NACA 0020 foil.}
\label{fig:blade-kinematics}
\end{figure}

Results for the peak torque coefficient for each foil are plotted versus $Re_c$
in Figure~\ref{fig:foils-C_T-Re-dep}. It is interesting that the convergence of
$C_{T_\mathrm{max}}$ with increasing Reynolds number is more dramatic than any of the
common foil characteristics plotted in Figure~\ref{fig:foil-Re-dep}, meaning
that the cross-flow turbine's unique kinematics must be taken into account when
attempting to predict the effect of transitional Reynolds numbers on turbine
performance. 

From the peak torque coefficient metric, the Reynolds number independence is
achieved at lower values, and more dramatically. We see that the trend of the
NACA 0020 curve matches almost perfectly at low $Re$, but the trend to continue
increasing linearly is not matched by the experimental data, which looks to
behave more like the cambered foil results. Though this method is very
approximate, it nevertheless appears to be a reasonable way to predict the
transitional Reynolds number regime for cross-flow turbine performance using
only 2-D static airfoil characteristics

\begin{figure}[ht!]
\centering
\includegraphics[width=0.65\textwidth]{figures/cft_re_dep_foils}
\caption{Reynolds number dependence of the peak torque coefficient calculated using 
only static foil coefficients and blade kinematics.}
\label{fig:foils-C_T-Re-dep}
\end{figure}


\subsection{Wake Characteristics}

\todo[inline]{Put wake characteristics here. Maybe create difference plots to
show absolute differences, or RMS, or something}


\todo[inline]{Plot asymptotic behavior for a bunch of locations all on the same
plot.}



\subsubsection{Dominant Timescales and Turbulence Spectra}

Spectra of the cross-stream velocity normalized by the free stream are plotted
in Figure~\ref{fig:wake-spectra} for regions on either side of the turbine. On
the $-y$ side of the turbine there is broadband turbulence produced by blade
stall, and on the $+y$ side there is a clear peak in the spectra caused by the
blade passage. We see that on both sides there is higher spectral energy at
lower Reynolds numbers. On the $+y$ side of the turbine we notice higher
spectral energy at the blade passage frequency's first harmonic, or $6
f_\mathrm{turbine}$.

\todo[inline]{Look at how spectra change when turbine rotational frequency
changes relative to inertial subrange.}

\todo[inline]{Make sure the spectra are not just shifting due to normalizing the
noise.}

\begin{figure}[ht!]
\centering
\includegraphics[width=0.95\textwidth]{figures/wake_spectra}
\caption{Cross-stream velocity (normalized by $U_\infty$) spectra at $z/H=0.25$, 
$y/R=-1.0$ (a) and $y/R=1.5$ (b). Dashed vertical
lines indicate $[1, 3, 6, 9]$ times the turbine rotational frequency.}
\label{fig:wake-spectra}
\end{figure}



\subsubsection{Transport of Mean Momentum and Kinetic Energy}

The relative importance of various physical processes on mean streamwise
momentum and kinetic energy transport/recovery in the streamwise direction are
plotted in Figures~\ref{fig:mom-bar-graph} and \ref{fig:K-bar-graph},
respectively. We note that similar to \cite{Bachant2015-JoT}, the vertical
advection at this point in the wake is the dominant contributor to positive wake
recovery, caused by the unique vortex pattern created by the blade tips and
wakes.

We see that in general, levels of turbulent transport are slightly lower at
lower Reynolds numbers. The viscous diffusion and dissipation, though still
three orders of magnitude smaller than the other terms, do increase at low
Reynolds numbers, which is consistent with the definition of the Reynolds number
in general.

\todo[inline]{At what $Re$ will the viscous terms get large enough to start
messing with the balance of the equation?}

\begin{figure}[ht!]
\centering
\includegraphics[width=0.9\textwidth]{figures/mom_bar_graph}
\caption{Momentum transport quantities.}
\label{fig:mom-bar-graph}
\end{figure}

\begin{figure}[ht!]
\centering
\includegraphics[width=0.9\textwidth]{figures/K_trans_bar_graph}
\caption{Mean kinetic energy transport quantities.}
\label{fig:K-bar-graph}
\end{figure}


Totals for all the wake transport terms calculated are plotted in
Figure~\ref{fig:wake-trans-totals}. We see that in general wake transport is
enhanced at lower Reynolds numbers and levels off consistent with the behavior
of the  turbine power coefficient. This is an important consideration if
studying sub-scale models of turbine arrays, where increased levels of wake
recovery will motivate different ideal array configurations when compared with
full-scale turbines.


\begin{figure}[ht!]
\centering
\includegraphics[width=0.65\textwidth]{figures/wake_trans_totals}
\caption{Normalized transport totals from Figures~\ref{fig:mom-bar-graph} and 
\ref{fig:K-bar-graph} plotted versus Reynolds number.}
\label{fig:wake-trans-totals}
\end{figure}


\section{Conclusions}

In this study we determined that, as previously shown, turbine performance
becomes essentially $Re$-independent at a turbine diameter Reynolds number $Re_D
\approx 10^6$, or an approximate blade chord Reynolds number $Re_c \approx 2
\times 10^5$.

These results suggest that to validate predictive engineering models, one must
use data of at least this scale, especially high-fidelity CFD models, where
transition to turbulence may be completely different between scaled physical
model and prototype.

If using scaled physical models to predict array performance, it is important to
keep all turbines in the linear regime to avoid exaggerated power deficiencies
for downstream turbines, despite similarities in wake characteristics. These
results also suggest that low Reynolds number physical model stydies of turbine
arrays may see exaggerated levels of wake recovery, leading to inadequate or
inappropriate spacing or layouts.

\acknowledgements{Acknowledgements}

The authors would like to acknowledge funding through a National Science
Foundation CAREER award (PI Wosnik, NSF-CBET 1150797, program manager Dr. Ram
Gupta), a grant through the Leslie S. Hubbard Marine Program Endowment to
purchase acoustic flow measurement instrumentation, and a grant for laboratory
infrastructure upgrades through the US Department of Energy.

\conflictofinterests{Conflicts of Interest}

The authors declare no conflict of interest.


\bibliography{library}
\bibliographystyle{mdpi}

\end{document}


